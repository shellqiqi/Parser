\section{词法语法描述}
\subsection{C11规范}
C11规范来自于cppreference\footnote{http://en.cppreference.com} 中所列举的C语言结构描述。Mini-C从中选取了常用的关键字来构成 C 语言的子集,包括数字、常用的数据类型、条件控制语句、\verb|For|循环、和常见的运算操作符与函数调用。为了表示的简洁性,本次实验只保留条件控制语句。本次报告的示例程序如下所示。
{\linespread{1}
\begin{lstlisting}[caption=本次实验的实验用例]
if (i)
    if (j)
        a;
    else
        b;
else
    if (k)
        d;
\end{lstlisting}}
\subsection{Mini-C正规式描述}
参照C11规范,我们得到Mini-C的词法正规式。
{\linespread{1}
\begin{lstlisting}[caption=Mini-C的词法正规式]
delim       [ \t\n]
ws          {delim}+
digit       [0-9]
digits      {digit}+
number      {digits}(\.{digits})?([Ee][+-]?{digits})?
symbol      "("|")"|"{"|"}"|";"|"["|"]"|","
letter_     [A-Za-z\_]
id          {letter_}({letter_}|{digit})*
str         \"[^\"]*\"
comparison  "<"|">"|"<="|">="|"=="|"!="
operator    "+"|"-"|"*"|"/"|"="|"++"|"--"|"<<"|">>"|"||"|"&&"
\end{lstlisting}}
有一些简单的正则表达式模式,比如~\verb|if|。如果我们在输入中看到两个字母~\verb|if|,并且~\verb|if|之后没有跟随其他字母和数字,就会返回词法单元~\verb|IF|。这样的保留字我们用一个保留字表来维持,在获得词素之后参照保留字来判断词法单元是什么。其他关键字处理方式与之类似。
{\linespread{1}
\begin{lstlisting}[caption=Mini-C的关键字]
if else for void char int return
\end{lstlisting}}
\subsection{BNF范式}
下面给出Mini-C的BNF范式
\begin{align*}
    \text{S } & \rightarrow \text{ if B S}\\
    \text{S } & \rightarrow \text{ if B S else S}\\
    \text{S } & \rightarrow \text{ id ;}\\
    \text{B } & \rightarrow \text{ ( id )}
\end{align*}
\subsection{LR(1)项集}
详见第\pageref{fig:LR1}~页附录\ref{sec:LR1}~图\ref{fig:LR1}。
\subsection{LR(1)分析表}
详见第\pageref{fig:LR1table}~页附录\ref{sec:LR1table}~表\ref{fig:LR1table}。
